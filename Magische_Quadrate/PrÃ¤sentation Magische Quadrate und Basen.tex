\documentclass[11pt]{beamer}
\usepackage[utf8]{inputenc}
\usepackage[T1]{fontenc}
\usepackage{lmodern}
\usepackage[ngerman]{babel}
\usepackage{amsmath}
\usepackage{amsfonts}
\usepackage{amssymb}
\usepackage{graphicx}
\usetheme{Antibes}

\begin{document}
	\author{Matheschülerzirkel 2018/2019}
	\title{Magische Quadrate}
	%\subtitle{}
	%\logo{}
	%\institute{}
	%\date{}
	%\subject{}
	%\setbeamercovered{transparent}
	%\setbeamertemplate{navigation symbols}{}
\begin{frame}[plain]
	%\maketitle
	\begin{center}
		\includegraphics[width=0.6\textwidth]{Melencolia_I_(Durero)}
	\end{center}
\end{frame}

\begin{frame}
	\frametitle{}
	\begin{center}
		\includegraphics[width=0.7\textwidth]{albrecht-duerer-magisches-quadrat}
	\end{center}
\end{frame}

\begin{frame}
	\frametitle{}
		\underline{Frage: Wie kann man solche systematisch Quadrate erzeugen?}
	\begin{itemize}
		\item Forderung: Summe der Zeilen, Spalten und Diagonalen soll gleich sein
		\item Dimension $4\times4$
		\item zur Vereinfachung: nur mit 0 und 1 
	\end{itemize}
\end{frame}

\begin{frame}
\frametitle{}
\underline{Frage: Wie kann man solche systematisch Quadrate erzeugen?}
\begin{itemize}
	\item Forderung: Summe der Zeilen, Spalten und Diagonalen soll gleich sein
	\item Dimension $4\times4$
	\item zur Vereinfachung: nur mit 0 und 1 
\end{itemize}
\underline{triviale F\"alle}:\\
\[
\begin{array}{|cccc|}
\hline
0 & 0 & 0 & 0 \\ 
0 & 0 & 0 & 0 \\ 
0 & 0 & 0 & 0 \\ 
0 & 0 & 0 & 0 \\
\hline
\end{array} 
\qquad \qquad \qquad
\begin{array}{|cccc|}
\hline
1 & 1 & 1 & 1 \\ 
1 & 1 & 1 & 1 \\ 
1 & 1 & 1 & 1 \\ 
1 & 1 & 1 & 1 \\
\hline
\end{array} 
\]

\end{frame}

\begin{frame}
\frametitle{Aufgabe 1}
\begin{center}
Findet alle möglichen magischen Quadrate, deren Zeilen-, Spalten-, Diagonalsumme 1 ergibt.\\
Wie viele gibt es?
\end{center} 
\end{frame}

\begin{frame}
	\frametitle{Basen}
	
	 \begin{align*}
	\mathcal{Q} = 
	\begin{array}{|cccc|}
	\hline
	1 & 0 & 0 & 0 \\ 
	0 & 0 & 1 & 0 \\ 
	0 & 0 & 0 & 1 \\ 
	0 & 1 & 0 & 0 \\
	\hline
	\end{array} 
	\qquad
	\mathcal{R} = 
	\begin{array}{|cccc|}
	\hline
	0 & 0 & 0 & 1 \\ 
	1 & 0 & 0 & 0 \\ 
	0 & 0 & 1 & 0 \\ 
	0 & 1 & 0 & 0 \\
	\hline
	\end{array} 
	\qquad
	\mathcal{S} = 
	\begin{array}{|cccc|}
	\hline
	0 & 0 & 1 & 0 \\ 
	1 & 0 & 0 & 0 \\ 
	0 & 1 & 0 & 0 \\ 
	0 & 0 & 0 & 1 \\
	\hline
	\end{array}
	\\
	\mathcal{T} = 
	\begin{array}{|cccc|}
	\hline
	0 & 0 & 1 & 0 \\ 
	0 & 1 & 0 & 0 \\ 
	0 & 0 & 0 & 1 \\ 
	1 & 0 & 0 & 0 \\
	\hline
	\end{array} 
	\qquad
	\mathcal{U} = 
	\begin{array}{|cccc|}
	\hline
	1 & 0 & 0 & 0 \\ 
	0 & 0 & 0 & 1 \\ 
	0 & 1 & 0 & 0 \\ 
	0 & 0 & 1 & 0 \\
	\hline
	\end{array} 
	\qquad
	\mathcal{V} = 
	\begin{array}{|cccc|}
	\hline
	0 & 0 & 0 & 1 \\ 
	0 & 1 & 0 & 0 \\ 
	1 & 0 & 0 & 0 \\ 
	0 & 0 & 1 & 0 \\
	\hline
	\end{array}
	\\
	\mathcal{W} = 
	\begin{array}{|cccc|}
	\hline
	0 & 1 & 0 & 0 \\ 
	0 & 0 & 1 & 0 \\ 
	1 & 0 & 0 & 0 \\ 
	0 & 0 & 0 & 1 \\
	\hline
	\end{array}
	\qquad
	\mathcal{X} = 
	\begin{array}{|cccc|}
	\hline
	0 & 1 & 0 & 0 \\ 
	0 & 0 & 0 & 1 \\ 
	0 & 0 & 1 & 0 \\ 
	1 & 0 & 0 & 0 \\
	\hline
	\end{array}   
	\end{align*}	
\end{frame}

\begin{frame}
	\frametitle{Aufgabe 2}
	\begin{center}
		Finde die Basisquadrate, aus denen das magische Quadrat zusammengesetzt ist!
		
		\begin{align*}
		\mathcal{A} =
		\begin{array}{|cccc|}
		\hline
		2 & 0 & 7 & 3 \\ 
		7 & 3 & 2 & 0 \\ 
		3 & 7 & 0 & 2 \\ 
		0 & 2 & 3 & 7 \\
		\hline
		\end{array}  
		\end{align*}
	\end{center}
\end{frame}

\begin{frame}
\frametitle{Aufgabe 3}
\begin{center}
	
	Erg\"anze das untenstehende Quadrat so, dass die Summe der Zahlen alles Zeilen, Spalten und Diagonalen gleich ist. 
\end{center}
\begin{align*}
\mathcal{B} =
\begin{array}{|cccc|}
\hline
2 & 7 & 9 & 8 \\ 
& 6 &  &  \\ 
&  & 9 &  \\ 
&  & 7 &  \\
\hline
\end{array}  
\end{align*}
\end{frame}

\begin{frame}
\frametitle{Darstellung des Dürer-Quadrats}
\begin{align*}
\mathcal{D} =
\begin{array}{|cccc|}
\hline
16 & 3 & 2 & 13 \\ 
5 & 10 &  11& 8  \\ 
9 &  6 & 7 &  12\\ 
4 &  15 & 14 &  1\\
\hline
\end{array}  
\end{align*}
\[
 = \quad  11 * \mathcal{Q} + 9*\mathcal{B} + 5*\mathcal{U} + 4*\mathcal{R} + 3*\mathcal{X} + \mathcal{S} + \mathcal{T}
\]


\end{frame}

\end{document}